%%% Template originaly created by Karol Kozioł (mail@karol-koziol.net) and modified for ShareLaTeX use

\documentclass[a4paper,11pt]{article}

\usepackage[T1]{fontenc}
\usepackage[utf8]{inputenc}
\usepackage{graphicx}
\usepackage{xcolor}

\renewcommand\familydefault{\sfdefault}
\usepackage{tgheros}
\usepackage[defaultmono]{droidmono}

\usepackage{amsmath,amssymb,amsthm,textcomp}
\usepackage{enumerate}
\usepackage{multicol}
\usepackage{tikz}

\usepackage{geometry}
\geometry{left=25mm,right=25mm,%
bindingoffset=0mm, top=20mm,bottom=20mm}


\linespread{1.3}

\newcommand{\linia}{\rule{\linewidth}{0.5pt}}

% custom theorems if needed
\newtheoremstyle{mytheor}
    {1ex}{1ex}{\normalfont}{0pt}{\scshape}{.}{1ex}
    {{\thmname{#1 }}{\thmnumber{#2}}{\thmnote{ (#3)}}}

\theoremstyle{mytheor}
\newtheorem{defi}{Definition}

% my own titles
\makeatletter
\renewcommand{\maketitle}{
\begin{center}
\vspace{2ex}
{\huge \textsc{\@title}}
\vspace{1ex}
\\
\linia\\
\@author \hfill \@date
\vspace{4ex}
\end{center}
}
\makeatother
%%%

% custom footers and headers
\usepackage{fancyhdr}
\pagestyle{fancy}
\lhead{}
\chead{}
\rhead{}
\lfoot{Assignment 1}
\cfoot{Approximation and randomized algorithms}
\rfoot{Page \thepage}
\renewcommand{\headrulewidth}{0pt}
\renewcommand{\footrulewidth}{0pt}
%

% code listing settings
\usepackage{listings}
\lstset{
    language=Python,
    basicstyle=\ttfamily\small,
    aboveskip={1.0\baselineskip},
    belowskip={1.0\baselineskip},
    columns=fixed,
    extendedchars=true,
    breaklines=true,
    tabsize=4,
    prebreak=\raisebox{0ex}[0ex][0ex]{\ensuremath{\hookleftarrow}},
    frame=lines,
    showtabs=false,
    showspaces=false,
    showstringspaces=false,
    keywordstyle=\color[rgb]{0.627,0.126,0.941},
    commentstyle=\color[rgb]{0.133,0.545,0.133},
    stringstyle=\color[rgb]{01,0,0},
    numbers=left,
    numberstyle=\small,
    stepnumber=1,
    numbersep=10pt,
    captionpos=t,
    escapeinside={\%*}{*)}
}

%%%----------%%%----------%%%----------%%%----------%%%

\begin{document}

\title{Assignment 1}

\author{Matjaž Mav (63130148)}

\date{\today}

\maketitle

\section*{1. Reduction of the N-queens problem to SAT}
Code for the reduction of the N-queens to SAT problem is available in \textit{src/reduce\_nq\_sat.py}.

\section*{2. Reduction of the dominanting set problem to SAT}

\subsection*{Description}

Input: Graph $G = \{V, E\}$ and integer $k$.\\
Problem: Does in graph $G$ contains a dominanting set $\{DS\}$ of size $\geq k$?\\
Variables: $x_i^r$ is true if node $i$ is the $r$-th node of the $\{DS\}$, where $1 \leq i \leq n$, $1 \leq r \leq k$ and $n = |V|$.\\
Clauses:
\begin{enumerate}
  \item Some node is the $r$-th node of the $\{DS\}$:\\
        For each $r$: $x_1^r \lor x_2^r \lor ... \lor x_n^r$
  \item No node is both the $r$-th and $s$-th node of the $\{DS\}$:\\
        For each $i$, $r<s$: $\neg x_i^r \lor \neg x_i^s$
  \item Only one node can be the $r$-th node of the $\{DS\}$:\\
        For each $r$, $i<j$: $\neg x_i^r \lor \neg x_j^r$
  \item At least one of the $i$-th node or one of its neighbour nodes must be in the $\{DS\}$:\\
        For each $i$ and its neighbours\footnote{Node $adj(i, m)$ is the $m$-th neighbour of the $i$-th node.}: $(x_i^1 \lor x_{adj(i, 1)}^1 \lor x_{adj(i, 2)}^1 \lor ... \lor x_{adj(i, m)}^1) \lor ... \lor (x_i^r \lor x_{adj(i, 1)}^r \lor x_{adj(i, 2)}^r \lor ... \lor x_{adj(i, m)}^r)$
\end{enumerate}

\subsection*{Code}

Code for the reduction of the dominanting set to SAT problem is available in \textit{src/reduce\_ds\_sat.py}.

\newpage
\subsection*{Solutions}

We have manually performed search with bisection and found the following sizes of the minimal dominanting sets:

\begin{enumerate}
  \item \textit{g1.col}: 40
  \item \textit{g2.col}: 3
  \item \textit{g3.col}: 15
  \item \textit{g4.col}: Not found any solution
  \item \textit{g5.col}: 5
\end{enumerate}

\section*{3. Other}

We have also implemented reduction of the vertex cover (\textit{src/reduce\_vc\_sat.py}) and clique pro-blem (\texit{src/reduce\_clique\_sat.py}) the to SAT.

\end{document}